% Options for packages loaded elsewhere
\PassOptionsToPackage{unicode}{hyperref}
\PassOptionsToPackage{hyphens}{url}
%
\documentclass[
]{article}
\usepackage{amsmath,amssymb}
\usepackage{lmodern}
\usepackage{iftex}
\ifPDFTeX
  \usepackage[T1]{fontenc}
  \usepackage[utf8]{inputenc}
  \usepackage{textcomp} % provide euro and other symbols
\else % if luatex or xetex
  \usepackage{unicode-math}
  \defaultfontfeatures{Scale=MatchLowercase}
  \defaultfontfeatures[\rmfamily]{Ligatures=TeX,Scale=1}
\fi
% Use upquote if available, for straight quotes in verbatim environments
\IfFileExists{upquote.sty}{\usepackage{upquote}}{}
\IfFileExists{microtype.sty}{% use microtype if available
  \usepackage[]{microtype}
  \UseMicrotypeSet[protrusion]{basicmath} % disable protrusion for tt fonts
}{}
\makeatletter
\@ifundefined{KOMAClassName}{% if non-KOMA class
  \IfFileExists{parskip.sty}{%
    \usepackage{parskip}
  }{% else
    \setlength{\parindent}{0pt}
    \setlength{\parskip}{6pt plus 2pt minus 1pt}}
}{% if KOMA class
  \KOMAoptions{parskip=half}}
\makeatother
\usepackage{xcolor}
\usepackage[margin=1in]{geometry}
\usepackage{color}
\usepackage{fancyvrb}
\newcommand{\VerbBar}{|}
\newcommand{\VERB}{\Verb[commandchars=\\\{\}]}
\DefineVerbatimEnvironment{Highlighting}{Verbatim}{commandchars=\\\{\}}
% Add ',fontsize=\small' for more characters per line
\usepackage{framed}
\definecolor{shadecolor}{RGB}{248,248,248}
\newenvironment{Shaded}{\begin{snugshade}}{\end{snugshade}}
\newcommand{\AlertTok}[1]{\textcolor[rgb]{0.94,0.16,0.16}{#1}}
\newcommand{\AnnotationTok}[1]{\textcolor[rgb]{0.56,0.35,0.01}{\textbf{\textit{#1}}}}
\newcommand{\AttributeTok}[1]{\textcolor[rgb]{0.77,0.63,0.00}{#1}}
\newcommand{\BaseNTok}[1]{\textcolor[rgb]{0.00,0.00,0.81}{#1}}
\newcommand{\BuiltInTok}[1]{#1}
\newcommand{\CharTok}[1]{\textcolor[rgb]{0.31,0.60,0.02}{#1}}
\newcommand{\CommentTok}[1]{\textcolor[rgb]{0.56,0.35,0.01}{\textit{#1}}}
\newcommand{\CommentVarTok}[1]{\textcolor[rgb]{0.56,0.35,0.01}{\textbf{\textit{#1}}}}
\newcommand{\ConstantTok}[1]{\textcolor[rgb]{0.00,0.00,0.00}{#1}}
\newcommand{\ControlFlowTok}[1]{\textcolor[rgb]{0.13,0.29,0.53}{\textbf{#1}}}
\newcommand{\DataTypeTok}[1]{\textcolor[rgb]{0.13,0.29,0.53}{#1}}
\newcommand{\DecValTok}[1]{\textcolor[rgb]{0.00,0.00,0.81}{#1}}
\newcommand{\DocumentationTok}[1]{\textcolor[rgb]{0.56,0.35,0.01}{\textbf{\textit{#1}}}}
\newcommand{\ErrorTok}[1]{\textcolor[rgb]{0.64,0.00,0.00}{\textbf{#1}}}
\newcommand{\ExtensionTok}[1]{#1}
\newcommand{\FloatTok}[1]{\textcolor[rgb]{0.00,0.00,0.81}{#1}}
\newcommand{\FunctionTok}[1]{\textcolor[rgb]{0.00,0.00,0.00}{#1}}
\newcommand{\ImportTok}[1]{#1}
\newcommand{\InformationTok}[1]{\textcolor[rgb]{0.56,0.35,0.01}{\textbf{\textit{#1}}}}
\newcommand{\KeywordTok}[1]{\textcolor[rgb]{0.13,0.29,0.53}{\textbf{#1}}}
\newcommand{\NormalTok}[1]{#1}
\newcommand{\OperatorTok}[1]{\textcolor[rgb]{0.81,0.36,0.00}{\textbf{#1}}}
\newcommand{\OtherTok}[1]{\textcolor[rgb]{0.56,0.35,0.01}{#1}}
\newcommand{\PreprocessorTok}[1]{\textcolor[rgb]{0.56,0.35,0.01}{\textit{#1}}}
\newcommand{\RegionMarkerTok}[1]{#1}
\newcommand{\SpecialCharTok}[1]{\textcolor[rgb]{0.00,0.00,0.00}{#1}}
\newcommand{\SpecialStringTok}[1]{\textcolor[rgb]{0.31,0.60,0.02}{#1}}
\newcommand{\StringTok}[1]{\textcolor[rgb]{0.31,0.60,0.02}{#1}}
\newcommand{\VariableTok}[1]{\textcolor[rgb]{0.00,0.00,0.00}{#1}}
\newcommand{\VerbatimStringTok}[1]{\textcolor[rgb]{0.31,0.60,0.02}{#1}}
\newcommand{\WarningTok}[1]{\textcolor[rgb]{0.56,0.35,0.01}{\textbf{\textit{#1}}}}
\usepackage{graphicx}
\makeatletter
\def\maxwidth{\ifdim\Gin@nat@width>\linewidth\linewidth\else\Gin@nat@width\fi}
\def\maxheight{\ifdim\Gin@nat@height>\textheight\textheight\else\Gin@nat@height\fi}
\makeatother
% Scale images if necessary, so that they will not overflow the page
% margins by default, and it is still possible to overwrite the defaults
% using explicit options in \includegraphics[width, height, ...]{}
\setkeys{Gin}{width=\maxwidth,height=\maxheight,keepaspectratio}
% Set default figure placement to htbp
\makeatletter
\def\fps@figure{htbp}
\makeatother
\setlength{\emergencystretch}{3em} % prevent overfull lines
\providecommand{\tightlist}{%
  \setlength{\itemsep}{0pt}\setlength{\parskip}{0pt}}
\setcounter{secnumdepth}{-\maxdimen} % remove section numbering
\ifLuaTeX
  \usepackage{selnolig}  % disable illegal ligatures
\fi
\IfFileExists{bookmark.sty}{\usepackage{bookmark}}{\usepackage{hyperref}}
\IfFileExists{xurl.sty}{\usepackage{xurl}}{} % add URL line breaks if available
\urlstyle{same} % disable monospaced font for URLs
\hypersetup{
  hidelinks,
  pdfcreator={LaTeX via pandoc}}

\author{}
\date{\vspace{-2.5em}}

\begin{document}

\hypertarget{dose-response-generating-function}{%
\section{Dose-response generating
function}\label{dose-response-generating-function}}

\hypertarget{latex}{%
\subsection{Latex}\label{latex}}

\begin{align*}
\text{Response}(d) = \begin{cases}
100 - \left(\frac{d}{d_{max}}\right) \cdot E_{max} & \text{decreasing} \\[2ex]
100 - E_{max}\left(1 - \left(\frac{d - d_{mid}}{d_{mid}}\right)^2\right) & \text{non-monotonic} \\[2ex]
\begin{cases}
100 & \text{if } d < d_{threshold} \\
100 - \left(\frac{d - d_{threshold}}{d_{max} - d_{threshold}}\right) \cdot E_{max} & \text{if } d \geq d_{threshold}
\end{cases} & \text{threshold} \\[2ex]
100 + E_{max}\sin(2\pi f d) & \text{oscillating} \\[2ex]
100 & \text{none}
\end{cases}
\end{align*}

\hypertarget{markdown}{%
\subsection{Markdown}\label{markdown}}

The response function R(d) is defined as follows:

For decreasing response: R(d) = 100 - (d/d\_max) × E\_max

For non-monotonic response: R(d) = 100 - E\_max × (1 - ((d -
d\_mid)/d\_mid)²)

For threshold response: R(d) = 100 if d \textless{} d\_threshold R(d) =
100 - ((d - d\_threshold)/(d\_max - d\_threshold)) × E\_max if d ≥
d\_threshold

For oscillating response: R(d) = 100 + E\_max × sin(2πfd)

For no response: R(d) = 100

Where: - d is the dose - d\_max is the maximum dose - d\_mid is the mean
of the dose range - E\_max is the maximum effect - f is the frequency
(set to 1) - d\_threshold is the threshold dose

\hypertarget{variance-pattern-generation}{%
\section{Variance Pattern
Generation}\label{variance-pattern-generation}}

\begin{align*}
\sigma^2(i) = \begin{cases}
\sigma^2_b & \text{homogeneous} \\[2ex]
\sigma^2_b \cdot \left(0.5 + \frac{1.5(i-1)}{n-1}\right) & \text{increasing} \\[2ex]
\sigma^2_b \cdot \left(2 - \frac{1.5(i-1)}{n-1}\right) & \text{decreasing} \\[2ex]
\sigma^2_b \cdot \begin{cases}
2 - \frac{i-1}{\lceil n/2 \rceil-1} & \text{if } i \leq \lceil n/2 \rceil \\[1ex]
1 + \frac{i-\lceil n/2 \rceil}{n-\lceil n/2 \rceil} & \text{if } i > \lceil n/2 \rceil
\end{cases} & \text{v-shaped}
\end{cases}
\end{align*}

The variance σ²(i) at dose level i is defined as follows:

For homogeneous variance: σ²(i) = σ²\_b

For increasing variance: σ²(i) = σ²\_b × (0.5 + 1.5(i-1)/(n-1))

For decreasing variance: σ²(i) = σ²\_b × (2 - 1.5(i-1)/(n-1))

For v-shaped variance: σ²(i) = σ²\_b × (2 - (i-1)/⌈n/2⌉-1) for i ≤ ⌈n/2⌉
σ²(i) = σ²\_b × (1 + (i-⌈n/2⌉)/(n-⌈n/2⌉)) for i \textgreater{} ⌈n/2⌉

Where: - i is the dose level index (1 to n) - n is the number of dose
levels - σ²\_b is the base variance (default = 4) - ⌈n/2⌉ represents the
ceiling of n/2 (mid-point)

\textbf{Markdown Version:}

\begin{Shaded}
\begin{Highlighting}[]
\NormalTok{The variance σ²(i) at dose level i is defined as follows:}

\NormalTok{For homogeneous variance:}
\NormalTok{σ²(i) = σ²\_b}

\NormalTok{For increasing variance:}
\NormalTok{σ²(i) = σ²\_b × (0.5 + 1.5(i{-}1)/(n{-}1))}

\NormalTok{For decreasing variance:}
\NormalTok{σ²(i) = σ²\_b × (2 {-} 1.5(i{-}1)/(n{-}1))}

\NormalTok{For v{-}shaped variance:}
\NormalTok{σ²(i) = σ²\_b × (2 {-} (i{-}1)/⌈n/2⌉{-}1)     for i ≤ ⌈n/2⌉}
\NormalTok{σ²(i) = σ²\_b × (1 + (i{-}⌈n/2⌉)/(n{-}⌈n/2⌉))   for i \textgreater{} ⌈n/2⌉}

\NormalTok{Where:}
\SpecialStringTok{{-} }\NormalTok{i is the dose level index (1 to n)}
\SpecialStringTok{{-} }\NormalTok{n is the number of dose levels}
\SpecialStringTok{{-} }\NormalTok{σ²\_b is the base variance (default = 4)}
\SpecialStringTok{{-} }\NormalTok{⌈n/2⌉ represents the ceiling of n/2 (mid{-}point)}
\end{Highlighting}
\end{Shaded}

These expressions describe how the variance changes across dose levels
for each variance pattern: - Homogeneous: Constant variance across all
doses - Increasing: Linear increase from 0.5×base\_var to 2×base\_var -
Decreasing: Linear decrease from 2×base\_var to 0.5×base\_var -
V-shaped: Decreases to base\_var at the midpoint, then increases back to
2×base\_var

Note that the actual implementation uses \texttt{seq()} for smooth
transitions, but these formulas represent the mathematical relationship
between dose level and variance.

\end{document}
